\documentclass[a4paper]{article}
\usepackage[
	pdftex,
	colorlinks=true,
	bookmarksnumbered=true,
	bookmarksopen=true,
	bookmarksopenlevel=3,
	pdfstartview=FitP,
	urlcolor=blue,
]{hyperref}
\pdfinfo{
	/Title(Esercitazione di Laboratorio: Misure su amplificatori)
	/Author(Coa Giulio, Licastro Dario, Montano Alessandra)
}
\usepackage[italian]{babel}
\usepackage{geometry,titling,mdsymbol,stmaryrd,graphicx,subcaption,amsmath}
\graphicspath{{./Image/}}
\renewcommand\maketitlehooka{
	\null
	\mbox{}
	\vfill
}
\renewcommand\maketitlehookd{
	\vfill
	\null
}
\title{
	\begin{center}
		Esercitazione di Laboratorio:
	\end{center}
	\newline
	\begin{center}
		Misure su amplificatori
	\end{center}
}
\author{
	Coa Giulio
	\and
	Licastro Dario
	\and
	Montano Alessandra
}
\begin{document}
	%-----------------------------------------------------------------------------
	%  TITLE
	%-----------------------------------------------------------------------------
	\begin{titlingpage}
		\maketitle
	\end{titlingpage}
	\newpage
	%-----------------------------------------------------------------------------
	%  PURPOSE OF THE EXPERIENCE
	%-----------------------------------------------------------------------------
	\section{Scopo dell'esperienza}
		Gli scopi di questa esercitazione sono:
		\begin{itemize}
			\item Analizzare il comportamento e misurare i parametri di moduli amplificatori (invertenti e non).
			\item Verificare alcune deviazioni rispetto al comportamento previsto con i modelli di prima approssimazione.
		\end{itemize}
	%-----------------------------------------------------------------------------
	%  INSTRUMENTATION USED
	%-----------------------------------------------------------------------------
	\section{Strumentazione utilizzata}
		La strumentazione usata durante l'esercitazione è:
		\begin{center}
			\begin{tabular}{ |c|c|c| }
				\multirow{Multimetro}			 & Agilent 34401A			& \\
				\multirow{Oscilloscopio}		 & Rigol DS1054Z			& 4 canali, \\
												 &							& $ B = 50 \, \mathrm{MHz} $, \\
												 &							& $ f_{\mathrm{c}} = 1 \, \mathrm{G\frac{Sa}{s}} $, \\
												 &							& $ R_{\mathrm{i}} = 1 \, \mathrm{M\Omega} $, \\
												 &							& $ C_{\mathrm{i}} = 13 \, \mathrm{pF} $, \\
												 &							& $ 12 \, \mathrm{Mbps} $ di profondità di memoria \\
				\multirow{Generatore di segnali} & Rigol DG1022				& 2 canali, \\
												 &							& $ f_{\mathrm{uscita}} = 20 \, \mathrm{MHz} $, \\
												 &							& $ Z_{\mathrm{uscita}} = 50 \, \mathrm{\Omega} $ \\
				\multirow{Alimentatore in DC}	 & Rigol DP832				& 2 canali, \\
												 &							& $ f_{\mathrm{uscita}} = 20 \, \mathrm{MHz} $, \\
												 &							& $ Z_{\mathrm{uscita}} = 50 \, \mathrm{\Omega} $ \\
				\multirow{Sonda}				 & Rigol PVP215				& $ B = 35 \, \mathrm{MHz} $, \\
												 &							& $ V_{\mathrm{nominale}} = 300 \, \mathrm{V} $, \\
												 &							& $ L_{\mathrm{cavo}} = 1.2 \, \mathrm{m} $, \\
												 &							& $ R_{\mathrm{s}} = 1 \, \mathrm{M\Omega} $, \\
												 &							& Intervallo di compensazione: $ 10 \div 25 \, \mathrm{pF} $ \\
				\multirow{Scheda premontata}	 & A2						& \\
				\multirow{Cavi coassiali}		 &							& Capacità dell'ordine dei $ 80 \div 100 \, \mathrm{p\frac{F}{m}} $ \\
				\multirow{Connettori}			 &							& \\
				\hline
			\end{tabular}
		\end{center}
	%-----------------------------------------------------------------------------
	%  THEORETICAL PREMISES
	%-----------------------------------------------------------------------------
	\section{Premesse teoriche}
		\subsection{Incertezza sulla misura dell'oscilloscopio}
			La misura del valore di un segnale tramite l’oscilloscopio (sia esso l'ampiezza, la frequenza, il periodo, etc.) presenta un'incertezza che dipende, principalmente, da due fattori:
			\begin{itemize}
				\item l’incertezza strumentale introdotta dall’oscilloscopio (ricavabile dal manuale).
				\item l’incertezza di lettura dovuta all’errore del posizionamento dei cursori.
			\end{itemize}
			Quest’ultima incertezza deriva dal fatto che il segnale visualizzato non ha uno spessore nullo sullo schermo.
		\subsection{Other}
			.
	%-----------------------------------------------------------------------------
	%  LABORATORY EXPERIENCE
	%-----------------------------------------------------------------------------
	\section{Esperienza in laboratorio}
		In questa esercitazione ci viene richiesta la misura del guadagno, il calcolo della resistenza i ingresso e in uscita dei due amplificatori, e per quello non-invertente anche la risposta in frequenza con celle RC esterne (che vuol dire?)
		Per fare ciò abbiamo lavorato con una basetta premontana sulla quale, modificando i vari interruttori si aggingevano/toglievano al circuito varie resistenze.
		S1 in particolare è l'interruttore che determina se stiamo lavorando con un amplificatore invertente o con uno non-invertente.
		Il generatore di segnali viene collegato alla basetta tramite il connettore J1.
		L'alimentatore duale ... connesso su J8
		Abbiamo connesso all'oscilloscopio, tramite due BNC-coccodrillo, ingresso (J2 e J5 sulla basetta)
		e uscita (J6 e J7 sulla basetta) del circuito premontato sulla basetta. %dovremmo fare una foto
		In seguito verranno riportate rappresentazioni dei vari circuiti analizzati e misurati.
		\subsection{Amplificatore non-invertente}
			\subsubsection{Misura del guadagno}
				Abbiamo disposto gli interruttori come da istruzioni, ottenendo un segnale passante solo per l'amplificatore, e quindi gli elementi circuitali che gi appartengono. %riportare tabella
				Impostando frequenza ed ampiezza picco-picco come richiesto abbiamo misurato tramite cursori l'ampiezza d'ingresso e di uscita rappresentate dall'oscilloscopio, per poi calcolare Av = Vu/Vi.
				In seguito abbiamoeseguito i cacoli per diversi valori di frequenza.
			\subsubsection{Misura della resistenza equivalente in ingresso}
				Per misurare Ri, la resistenza all'ingresso dell'amplificatore ci avvaliamo di una resistenza esterna, di valore noto, mettendola in serie al generatore.
				Frequenza e ampiezza picco-picco restano invariate rispetto al punto precedente. 
				Nel concreto questo viene otenuto modificando la  posizione dello switch S5, che, se aperto, fa passare la corrente nella resistenza R9, se chiuso la cortocicuita.
				Abbiamo effettuato misurazioni sulla tensione in uscita sia con che senza R9, notando come la tensione in uscita ali quando la resisteza è connessa, usare Vu invece di Vi permette di evidenziare maggiormente quanto la resistenza influenzi il segnale. %riportare calcoli effettuari per ottenere Ri
				%pezzo da mettere in una parte più teorica.
				Le misurazioni sono state effettuate con l'osclilloscopio, quindi per mezzo dei cursori.
			\subsubsection{Misura della resistenza equivalente in uscita}	
				Per misurare Ru, la resistenza all'uscita dell'amplificatore ci avvaliamo di una resistenza esterna R10, di valore noto, mettendola in parallelo all'uscita.
				Frequenza e ampiezza picco-picco restano invariate rispetto al punto precedente. 
				Nel concreto questo viene otenuto modificando la  posizione dello switch S6, che, se chiuso, collega R10, se aperto non vi permette il passaggio di corrente.
				Abbiamo effettuato misurazioni sulla tensione in uscita sia con che senza R10, notando come la tensione in uscita cali quando la resisteza è connessa. %riportare calcoli effettuari per ottenere Ru
				Le misurazioni sono state effettuate con l'osclilloscopio, quindi per mezzo dei cursori.
				\subsubsection{Calcolo poli e zero}
					%homework matlab? 
		\subsection{Amplificatore invertente}
			Abbiamo commutato S1 ed S2 in modo che fosse connesso l'amplificatore invertente ed abbiamo effettuato le stesse misurazioni del punto precedente.
			\subsubsection{Misura del guadagno}
				Abbiamo disposto gli interruttori come da istruzioni, ottenendo un segnale passante solo per l'amplificatore, e quindi gli elementi circuitali che gi appartengono. %riportare tabella
				Impostando frequenza ed ampiezza picco-picco come richiesto abbiamo misurato tramite cursori l'ampiezza d'ingresso e di uscita rappresentate dall'oscilloscopio, per poi calcolare Av = Vu/Vi.
				Abbiamo poi effettuato il guadagno per una frequenza di 1 KHz.
			\subsubsection{Misura della resistenza equivalente in ingresso}	
				Per misurare Ri, la resistenza all'ingresso dell'amplificatore ci avvaliamo di una resistenza esterna, di valore noto, mettendola in serie al generatore.
				Nel concreto questo viene otenuto modificando la  posizione dello switch S5, che, se aperto, fa passare la corrente nella resistenza R9, se chiuso la cortocicuita.
				Abbiamo effettuato misurazioni sulla tensione in uscita sia con che senza R9, notando come la tensione in uscita ali quando la resisteza è connessa, usare Vu invece di Vi permette di evidenziare maggiormente quanto la resistenza influenzi il segnale. %riportare calcoli effettuari per ottenere Ri
				%pezzo da mettere in una parte più teorica.
				Le misurazioni sono state effettuate con l'osclilloscopio, quindi per mezzo dei cursori.
			\subsubsection{Misura della resistenza equivalente in uscita}	
				Per misurare Ru, la resistenza all'uscita dell'amplificatore ci avvaliamo di una resistenza esterna R10, di valore noto, mettendola in parallelo all'uscita.
				Nel concreto questo viene otenuto modificando la  posizione dello switch S6, che, se chiuso, collega R10, se aperto non vi permette il passaggio di corrente.
				Abbiamo effettuato misurazioni sulla tensione in uscita sia con che senza R10, questa volta la connessione della resistenza non infuenza la tensione in uscita, deduciamo che è trascurabile. %riportare calcoli effettuari per ottenere Ru
				Le misurazioni sono state effettuate con l'osclilloscopio, quindi per mezzo dei cursori.
	%-----------------------------------------------------------------------------
	%  RESULTS
	%-----------------------------------------------------------------------------
	\section{Risultati}
		\subsection{Amplificatore non-invertente}
			\subsubsection{Misura del guadagno}
				Alla fraquenza 0.8 kHz e Vpp = 1V abbiamo misurato Vi = 1.12 V e Vu = 8.72 V, di conseguenza Av = 7.78, Av = 17.82 dB. 
				Evidenziamo come l'amplificatore abbia amplificato il segnale ricevuto come input.
				%foto1
			\subsubsection{Misura della resistenza equivalente in ingresso}	
				R9 = 10 kOhm.
				Con R9 connessa Vu = 4.64 V, con R9 disconnessa Vu = 8.72 V
				%rispettivamente foto 2 e foto 3
			\subsubsection{Misura della resistenza equivalente in uscita}
				R10 = 1kOhm.
				Con R10 connessa Vu = 4.40 V, con R10 disconnessa Vu = 8.72 V
				%rispettivamente foto 4 e foto 3
		\subsection{Amplificatore invertente}
			Rispetto al non-invertente possiamo notare come Vs e Vu risultino invertite. 
			%foto 6
			\subsubsection{Misura del guadagno}
				Alla fraquenza 0.8 kHz e Vpp = 1V abbiamo misurato Vi = 1.12 V e Vu = 10.30 V, di conseguenza Av = 9.20, Av = 19.27 dB. 
				%foto6
				Abbiamo effettuato la misurazione anche per f = 1kHz:
				Vi = 1.08 V e Vu = 10.3 V, quindi Av= 9.53, Av = 19.59 dB
				%foto 9
			\subsubsection{Misura della resistenza equivalente in ingresso}	
				R9 = 10 kOhm.
				Con R9 connessa Vu = 6.24 V, con R9 disconnessa Vu = 10.3 V
				%rispettivamente foto7 e foto 8
			\subsubsection{Misura della resistenza equivalente in uscita}	
				R10 = 1kOhm.
				Con R10 connessa Vu = 10.3 V, con R10 disconnessa Vu = 10.3 V.
				Notiamo, quindi, che la resistenza non influenza la tensione in uscita.
				%rispettivamente foto 8 e foto 8
		\subsection{Other}
			.
\end{document}